\chapter{Considerações Finais}

Este trabalho, por meio do desenvolvimento de uma aplicação web denominada LabInstru Web, colabora na minimização dos problemas enfrentados pelos pesquisadores do LabInstru na realização das atribuições deste laboratório. Considerando o tempo disponível, a existência de um único desenvolvedor e a possibilidade do surgimento de novos requisitos a qualquer momento, o Processo Ágil Unificado é o processo de desenvolvimento que tem guiado as disciplinas que levaram ao desenvolvimento da aplicação até o seu estágio atual.

Na implementação do LabInstru Web foi utilizado o \emph{framework} Web2py como base para o desenvolvimento da aplicação. Para o \emph{front-end} as tecnologias ágeis JQuery e Bootstrap colaboraram para produzir resultados mais visualmente agradáveis junto aos usuários.  Quanto ao \emph{back-end}, o sistema gerenciador de banco de dados MySQL foi utilizado para persistência dos dados por prover um bom desempenho na manipulação de uma quantidade significativa de dados.

Tão logo as funcionalidades remanescentes sejam implementadas, almeja-se que, após a implantação, o LabInstru Web proporcione benefícios aos pesquisadores do LabInstru, auxiliando-os a organizarem os dados da estação meteorológica da EST, permitindo que possam consultar estes dados para suas pesquisas (iniciação científica, mestrado, doutorado, etc.) de maneira mais rápida e eficiente e colaborando com a divulgação dos dados junto à população em geral por meio de boletins meteorológicos.

O desenvolvimento deste trabalho de conclusão de curso tem proporcionado a aplicação de conhecimentos obtidos nas disciplinas da grade do curso de Engenharia da Computação com vistas a resolver um problema real de um laboratório de instrumentação meteorológica. Disciplinas como Engenharia de Software, Banco de Dados e Linguagem de Programação foram cruciais para o desenvolvimento do mesmo. Entretanto, alguns conhecimentos tiveram de ser adquiridos de maneira independente, a exemplo do \emph{framework} considerado no desenvolvimento da aplicação web aqui proposta.