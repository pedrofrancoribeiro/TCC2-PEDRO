As tecnologias utilizadas para elaboração da solução proposta consistem no \emph{framework} Web2py, detalhado anteriormente na Seção \ref{sec:web2py}, nos \emph{frameworks} Bootstrap, na biblioteca JQuery e no sistema gerenciador de banco de dados MySQL. Uma visão geral de cada uma dessas tecnologias será apresentado a seguir.

O Bootstrap é um \emph{framework} JavaScript, HTML e CSS para desenvolvimento de sites e aplicações web responsivas, possibilitando uma maior agilidade e facilidade no desenvolvimento do \emph{front-end} \cite{Silva:Livro}. Embora o Web2py utilize internamente este \emph{framework} para geração das \emph{views}, a utilização direta do Bootstrap no contexto deste trabalho foi necessária para proporcionar uma melhor customização das páginas web da  aplicação, resultando em um melhor dominio sobre as funcionalidades envolvidas nas páginas web.

O JQuery é uma biblioteca JavaScript que simplifica a manipulação de documentos HTML, eventos, animações e interações AJAX no desenvolvimento rápido de aplicações web \cite{Duckett:JS}. Assim como no caso do Bootstrap, o Web2py também faz uso interno desta biblioteca, porém a manipulação direta da mesma provê  uma melhor customização e controle das funcionalidades, considerando a adição de efeitos visuais, melhoria de aspectos de interatividade e simplificação de determinadas tarefas, razão pela qual considerou-se também a demanda por JQuery no contexto deste trabalho.

Como mencionado anteriormente, o \emph{framework} Web2py já possui o banco de dados SQLite integrado. Porém, este banco de dados possui algumas limitações de desempenho, razão pela qual optou-se pela adoção do MySQL, um sistema gerenciador de banco de dados considerado versátil e com suporte a diversas plataformas e diferentes linguagens, bastante utilizado em aplicações web e desktop \cite{MySql:MySql}. Para facilitar a administração do banco de dados, a ferramenta MySQL Workbench \cite{MySql:WorkBench} também foi utilizada, visando prover auxílio na visualização dos dados do banco, realizar testes e gerenciar as mudanças.
