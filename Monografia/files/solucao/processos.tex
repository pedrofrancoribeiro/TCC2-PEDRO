O AUP foi escolhido como modelo de processo de desenvolvimento por seguir a metodologia ágil e por possuir diversos elementos que capturam a realidade do contexto em que este trabalho de conclusão de curso está sendo desenvolvido. Além dos elementos mencionados, um outro fator prepoderante para escolha do AUP foi o fato do software  desenvolvido não ser de grande porte e possuir uma equipe pequena de desenvolvimento, nesse caso, composta apenas de uma pessoa -- o aluno.

A Profa. Maria Betânia Leal exerceu o papel de cliente final da aplicação. O papel de solicitante do software foi exercido pela Profa. Elloá B. Guedes, a qual forneceu \emph{feedback} e auxiliou na validação das funcionalidades desenvolvidas.

Por estar baseado em uma abordagem iterativa e incremental, o AUP auxiliou a organização deste trabalho em séries de pequenas iterações, que puderam ser efetuadas levando em consideração um caso de uso por vez. Desta maneira, teve-se sempre um sistema parcial executável e testável, em que as partes desenvolvidas puderam ser facilmente integradas.

Em relação à algumas disciplinas do AUP, algumas decisões foram consideradas. Em termos de testes, foram considerados os testes feitos pelo próprio desenvolvedor utilizando os recursos disponíveis na linguagem de programação adotada, tal como o comando \texttt{assert}. O gerenciamento de configuração será efetuado com o auxílio das ferramentas Google Drive e Git para gerenciamento de documentos e de código, respectivamente. A disciplina de gerenciamento de projetos foi liderada pela profa. Elloá B. Guedes, que promoveu reuniões com a cliente sempre que necessário e estabeleceu prazos, atividades e marcos de entrega de acordo com o planejamento semestral do trabalho de conclusão de curso.
