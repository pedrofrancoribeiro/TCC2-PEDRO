\chapter{Solução Proposta} \label{cap:solucao}

Em resposta aos problemas identificados para processar as informações geradas pela estação meteorológica da EST e geração dos boletins meteorológicos, este trabalho se propôs a projetar e implementar uma plataforma web -- denominada \emph{LabInstru Web} -- para gerenciamento de dados e geração de boletins meteorológicos do LabInstru. Esta plataforma, implementada com o \emph{framework full-stack} Web2py, banco de dados MySQL, e tecnologias como Bootstrap e JQuery, permite a manutenção dos dados oriundos da Estação Meteorológica da EST, possibilitando a consulta de calendários de disponibilidade, geração de gráficos e produção de um boletim meteorológico detalhado, contendo as informações coletadas ao longo de uma dia pela estação, índice de calor e rajada de vento.

Para apresentar a solução proposta, este capítulo está organizado como segue. O processo de desenvolvimento adotado é detalhado na Seção \ref{sec:processoAdotado}. Os elementos da modelagem, incluindo os diagramas de casos de uso, modelos conceitual e de entidade-relacionamento bem como a prototipação da interface gráfica com o usuário, são apresentados nas Seções \ref{sec:casoUso}-\ref{sec:prototipo}. A descrição das tecnologias utilizadas pode ser consultada na Seção \ref{sec:tecnologias}. A apresentação da solução proposta pode ser vista na Seção \ref{sec:apresenta}, em particular, com  detalhamentos para a geração de boletins meteorológicos, discussão sobre a não-implantação de uma das funcionalidades previamente mapeadas, e algumas métricas derivadas da solução proposta. Por fim, o ambiente de implantação da solução é mostrado na Seção \ref{sec:implantancao}.


\section{Processo de Desenvolvimento Adotado} \label{sec:processoAdotado}
O AUP foi escolhido como modelo de processo de desenvolvimento por seguir a metodologia ágil e por possuir diversos elementos que capturam a realidade do contexto em que este trabalho de conclusão de curso está sendo desenvolvido. Além dos elementos mencionados, um outro fator prepoderante para escolha do AUP foi o fato do software  desenvolvido não ser de grande porte e possuir uma equipe pequena de desenvolvimento, nesse caso, composta apenas de uma pessoa -- o aluno.

A Profa. Maria Betânia Leal exerceu o papel de cliente final da aplicação. O papel de solicitante do software foi exercido pela Profa. Elloá B. Guedes, a qual forneceu \emph{feedback} e auxiliou na validação das funcionalidades desenvolvidas.

Por estar baseado em uma abordagem iterativa e incremental, o AUP auxiliou a organização deste trabalho em séries de pequenas iterações, que puderam ser efetuadas levando em consideração um caso de uso por vez. Desta maneira, teve-se sempre um sistema parcial executável e testável, em que as partes desenvolvidas puderam ser facilmente integradas.

Em relação à algumas disciplinas do AUP, algumas decisões foram consideradas. Em termos de testes, foram considerados os testes feitos pelo próprio desenvolvedor utilizando os recursos disponíveis na linguagem de programação adotada, tal como o comando \texttt{assert}. O gerenciamento de configuração será efetuado com o auxílio das ferramentas Google Drive e Git para gerenciamento de documentos e de código, respectivamente. A disciplina de gerenciamento de projetos foi liderada pela profa. Elloá B. Guedes, que promoveu reuniões com a cliente sempre que necessário e estabeleceu prazos, atividades e marcos de entrega de acordo com o planejamento semestral do trabalho de conclusão de curso.


\section{Diagramas de Caso de Uso} \label{sec:casoUso}
Após conversas com a cliente e com a solicitante do software, foi possível identificar quatro módulos principais que contemplam as diferentes funcionalidades solicitadas. Estes módulos são apresentados a seguir:

\begin{enumerate}
	\item \textbf{Módulo Gerencia Conta de Usuário}. Neste módulo, cujo ator principal é o administrador do sistema, concentram-se as funcionalidades relativas à manutenção de usuários, ilustradas no diagrama de caso de uso da Figura \ref{fig:casoUso1}. No cenário em que a plataforma será utilizada, foi possível identificar que não deve haver livre cadastro e acesso aos dados de maneira deliberada, daí a necessidade de um administrador para cadastrar novos usuários, que podem ser alunos de iniciação científica, pesquisadores, docentes, dentre outros;
	\begin{figure}[H]
		\centering
		\includegraphics[scale=0.8]{img/uc001.png}
		\caption{Caso de Uso - Módulo Gerencia Conta de Usuário.}
		\label{fig:casoUso1}
	\end{figure}
	\item \textbf{Módulo Usuário}. Neste módulo, cujo ator principal é o usuário, encontram-se as funcionalidades relativas à manutenção dos dados do próprio usuário. Pode haver alterações de dados do perfil, redefinição e recuperação de senha, conforme ilustrado na Figura \ref{fig:casoUso2};
	\begin{figure}[H]
		\centering
		\includegraphics[scale=0.8]{img/uc002.png}
		\caption{Caso de Uso - Módulo Usuário.}
		\label{fig:casoUso2}
	\end{figure}
	\item \textbf{Módulo Consulta Medições}. Este módulo concentra as funcionalidades de acesso aos dados das estações. Conforme ilustra a Figura \ref{fig:casoUso3}, podem ser feitas consultas diretas aos dados, consulta à disponibilidade dos dados em um determinado mês e também aspectos da visualização dos dados, seja por meio de boletins como também por meio de gráficos, além da possibilidade de exportação. Estas funcionalidades foram identificadas considerando as principais solicitações de dados feitas por terceiros ao LabInstru;
	% figura 3
	\begin{figure}[H]
		\centering
		\includegraphics[scale=0.8]{img/uc003.png}
		\caption{Caso de Uso - Módulo Consulta Medições.}
		\label{fig:casoUso3}
	\end{figure}
	\item \textbf{Módulo Gerencia Medições}. O módulo de gerenciamento de medições permite que o administrador do sistema mantenha as medições do sistema a partir dos dados obtidos das estações meteorológicas do LabInstru. Considerando a importância de assegurar a origem destes dados e a remoção das eventuais medições inconsistentes, apenas o administrador está habilitado para execução das funcionalidades deste módulo, conforme ilustrado na Figura \ref{fig:casoUso4}.
	% figura 4
	\begin{figure}[H]
		\centering
		\includegraphics[scale=0.8]{img/uc004.png}
		\caption{Caso de Uso - Módulo Gerencia Medições.}
		\label{fig:casoUso4}
	\end{figure}
\end{enumerate}

O detalhamento de todos os casos de uso ilustrados nas Figuras \ref{fig:casoUso1}-\ref{fig:casoUso4} encontra-se disponível no Apêndice \ref{sec:aprendiceCasoUso},  onde podem ser vistos os interessados, pré e pós condições, fluxo principal, fluxo alternativo e regras de negócio.


\section{Modelo Conceitual}
Um modelo de sistema prima intencionalmente pela abstração simplificada da descrição do sistema a ser desenvolvido, ressaltando as caracterísiticas mais importantes que o mesmo pode apresentar. O modelo conceitual é um modelo de sistema simples, usado para representar os conceitos em um domínio do problema. É um modelo de arquitetura de alto nível, geralmente expresso como diagramas de blocos simples, nos quais cada conceito é representado por um retângulo identificado, com linhas indicando as associações entre seus subsistemas \cite{Sommerville:Livro}.

O modelo conceitual também pode ser utilizado para representar a modelagem entidade-relacionamento de banco de dados relacionais. Nesse contexto, entidades do MER podem ser associadas como classes simplificadas de objetos (sem possuirem suas operações), atributos com atributos de classes e nas ralações como associações identificadas entre as classes \cite{Sommerville:Livro}. Na Figura \ref{fig:modeloConceitual}, podemos verificar como esse modelo de sistema é aplicado ao domínio do problema do LabInstru Web.

\begin{figure}[H]
	\centering
	\includegraphics[scale=0.8]{img/ModeloConceitual.png}
	\caption{Módelo Conceitual - LabInstru Web.}
	\label{fig:modeloConceitual}
\end{figure}

No diagrama acima, podemos verificar que a entidade Usuário representa os atores responsáveis por interagir com a aplicação, o mesmo pode ser de um dos dois tipos: Admin ou User. O perfil de usuário Admin, possui privilégios especiais, como exemplo, ser apenas esse perfil capaz de alimentar a base de dados da aplicação com dados provenientes da estação meteorológica da est. Esse usuário administrador, também é o responsável em cadastrar e manter osusuários da plataforma web mencionada. O perfil de usuário User, poderá fazer sua autenticação por meio de um login, e utilizar as funcionalidades disponibilizadas pela aplicação, como as consultas às medições contidas no banco de dados da aplicação, exportação dessas consultas em um formato de dados específico (txt, csv, xml) ou exportação de gráficos referentes a essa determinada consulta. O Usuário User também pode solicitar a geração dos Boletins Meteorológicos, bem como os Calendários de Disponibilidades das medições persistidas na aplicação.

As entidades Baixa1 e Baixa2 representam as tabelas no banco de dados da aplicação, no qual, tais tabelas persistem os arquivos de texto no formato .dat gerados pela estação meteorológica da est. Essas duas entidades formam a base da aplicação web,  todas as consultas, exportações, gráficos, boletim meteorológico e calendário de disponibilidade são gerados a partir dessas bases de dados.

\section{Modelo Entidade-Relacionamento}
O principal objetivo da modelagem do \textbf{Modelo Entidade-Relacionamento (MER)}, que também é um modelo de dados conceitual de alto nível, é criar um \textbf{esquema conceitual} para o banco de dados da aplicação a ser desenvolvida. Os conceitos utilizados por esse modelo, incluem uma descrição concisa dos requisitos de dados dos usuários e inclui detalhes dos tipos de entidade, relacionamentos e restrições \cite{Navathe:Livro}.

Por não incluirem descrições detalhadas de implementação, esses conceitos normalmente são mais fáceis de entender e podem ser usados parra comunicação com usuários não técnicos. Porém, a maioria dos Sistemas Gerenciadores de Banco de Dados (SGBDs) comerciais utilizam um modelo de dados de implementação - como o modelo de banco de dados relacional ou objeto-relacional- de modo que o esquema conceitual é transformado do modelo de dados de alto nível para o modelo de dados da implementação, etapa conhecida como \textbf{projeto lógico} ou \textbf{mapeamento do modelo de dados} \cite{Navathe:Livro}.

\begin{figure}[H]
	\centering
	\includegraphics[scale=1.0]{img/mer.png}
	\caption{Módelo Entidade Relacionamento - LabInstru Web.}
	\label{fig:mer}
\end{figure}

na Figura \ref{fig:mer} é mostrado o Modelo Entidade Relaconamento que compõe o banco de dados da plataforma LabInstru Web. Visualmente ao compararmos o Modelo Conceitual e o MER,podemos verificar que ambos possuem semântica bastante similar.


\section{Prototipação das Telas do Usuário} \label{sec:prototipo}
Considerando as funcionalidades identificadas e documentadas, partiu-se para elaboração de protótipos. Estes protótipos foram construidos considerando as principais funcionalidades a serem desenvolvidas, com o intuito de mostrar à cliente uma representação limitada da solução proposta, mas que permitisse explorar a sua conveniência. O resultado desta prototipação é mostrado a seguir. Embora os protótipos de todas as funcionalidades tenham sido elaborados, apenas os mais relevantes serão mostrados a seguir. Para elaborá-los foi utilizado o software Balsamiq Mockups \cite{Prototipacao:Mockups}.

A tela inicial da aplicação encontra-se ilustrada na Figura  \ref{fig:tela002}. No canto superior direito, há um link para o formulário de autenticação no sistema e também para recuperação de senha, caso algum usuário tenha esquecido da mesma.

\begin{figure}[H]
	\centering
	\includegraphics[width=0.8\textwidth]{./img/telas/tela002.png}
	\caption{Tela inicial da aplicação web LabInstru.} \label{fig:tela002}
\end{figure}

Considerando a perspectiva do usuário Administrador, o menu principal da aplicação disponível para o mesmo é mostrado na Figura \ref{fig:tela025}, no qual é possível selecionar a opção ``Cadastrar Usuário'' na aba ``Administração''.  Para efetuar a inclusão de um novo usuário na base de dados, o administrador será redirecionado para o formulário de cadastro ilustrado na Figura \ref{fig:tela027}.


\begin{figure}[H]
	\centering
	\includegraphics[width=0.8\textwidth]{./img/telas/tela025.png}
	\caption{Protótipo de tela do menu principal da aplicação.} \label{fig:tela025}
\end{figure}


\begin{figure}[H]
	\centering
	\includegraphics[width=0.8\textwidth]{./img/telas/tela027.png}
	\caption{Protótipo de tela referente ao cadastrado de usuário.} \label{fig:tela027}
\end{figure}

Para exibe uma listagem dos usuários cadastrados na base de dados da aplicação, e posteriormente, caso desejável, fazer a edição ou remoção de um usuário específico, deve-se esoclher a opção ``Listar Usuários'' na aba ``Administração'' do menu principal da aplicação, conforme Figura \ref{fig:tela033}.

\begin{figure}[H]
	\centering
	\includegraphics[width=0.8\textwidth]{./img/telas/tela033.png}
	\caption{Protótipo de tela referente à listagem de usuários.} \label{fig:tela033}
\end{figure}

O cadastro de novas medições é efetuado pelo Administrador. Para tanto, este deve utilizar um formulário análogo ao mostrado na Figura \ref{fig:tela053}, em que este deve fornecer um arquivo oriundo da estação meteorológia no formato \texttt{.dat}.

\begin{figure}[H]
	\centering
	\includegraphics[width=0.8\textwidth]{./img/telas/tela053.png}
	\caption{Protótipo de tela para cadastro de novas medições.} \label{fig:tela053}
\end{figure}

Caso um usuário deseje efetuar uma consulta na base de dados, um formulário detalhado será exibido, conforme ilustrado na Figura  \ref{fig:tela058}, no qual o usuário deve informar os parâmetros para consulta dos dados. Ao submeter a consulta, as respostas serão exibidas conforme ilustrado na Figura \ref{fig:tela062}.

\begin{figure}[H]
	\centering
	\includegraphics[width=0.8\textwidth]{./img/telas/tela058.png}
	\caption{Protótipo de tela referente ao formulário para consultar medições.} \label{fig:tela058}
\end{figure}

\begin{figure}[H]
	\centering
	\includegraphics[width=0.8\textwidth]{./img/telas/tela062.png}
	\caption{Protótipo de tela referente à visão de saída a uma consulta de medições.} \label{fig:tela062}
\end{figure}

Por meio do menu principal da aplicação, escolhendo a opção disponibilidade, pertecente à aba de uma determinada estação meteorológica (Baixa 1 ou Baixa 2), vide Figura \ref{fig:tela072}, é possível ter acesso ao calendário de disponibilidade de medições diárias desta estação meteorológica. Esse calendário, conforme ilustrado na Figura \ref{fig:tela073}, tem por objetivo informar quantas medições estão disponíveis em todos os dias do mês escolhido, por meio de cores apropriadas.

\begin{figure}[H]
	\centering
	\includegraphics[width=0.8\textwidth]{./img/telas/tela072.png}
	\caption{Navegando no menu para funcionalidade Disponibilidade.} \label{fig:tela072}
\end{figure}

\begin{figure}[H]
	\centering
	\includegraphics[width=0.8\textwidth]{./img/telas/tela073.png}
	\caption{Tela responsável por exibir o calendário de disponibilidade.} \label{fig:tela073}
\end{figure}

Outra funcionalidade importante na aplicação é o \textit{Boletim Meteorológico}, que pode ser acessado por meio da opção \textit{``Boletim''}, na aba \textit{``Medições''} do menu principal da aplicação. Esse boletim informará diversos dados meteorológicos (temperatura máxima, mínima, índice de calor, etc.) por dia de um determinado mês. A Figura \ref{fig:tela077} ilustra um exemplo de processamento resultante desta funcionalidade.

\begin{figure}[H]
	\centering
	\includegraphics[width=0.8\textwidth]{./img/telas/tela077.png}
	\caption{Protótipo de tela responsável por mostrar o resultado do boletim meteorológico.} \label{fig:tela077}
\end{figure}


\section{Tecnologias Utilizadas} \label{sec:tecnologias}
As tecnologias utilizadas para elaboração da solução proposta consistem no \emph{framework} Web2py, detalhado anteriormente na Seção \ref{sec:web2py}, nos \emph{frameworks} Bootstrap, na biblioteca JQuery e no sistema gerenciador de banco de dados MySQL. Uma visão geral de cada uma dessas tecnologias será apresentado a seguir.

O Bootstrap é um \emph{framework} JavaScript, HTML e CSS para desenvolvimento de sites e aplicações web responsivas, possibilitando uma maior agilidade e facilidade no desenvolvimento do \emph{front-end} \cite{Silva:Livro}. Embora o Web2py utilize internamente este \emph{framework} para geração das \emph{views}, a utilização direta do Bootstrap no contexto deste trabalho foi necessária para proporcionar uma melhor customização das páginas web da  aplicação, resultando em um melhor dominio sobre as funcionalidades envolvidas nas páginas web.

O JQuery é uma biblioteca JavaScript que simplifica a manipulação de documentos HTML, eventos, animações e interações AJAX no desenvolvimento rápido de aplicações web \cite{Duckett:JS}. Assim como no caso do Bootstrap, o Web2py também faz uso interno desta biblioteca, porém a manipulação direta da mesma provê  uma melhor customização e controle das funcionalidades, considerando a adição de efeitos visuais, melhoria de aspectos de interatividade e simplificação de determinadas tarefas, razão pela qual considerou-se também a demanda por JQuery no contexto deste trabalho.

Como mencionado anteriormente, o \emph{framework} Web2py já possui o banco de dados SQLite integrado. Porém, este banco de dados possui algumas limitações de desempenho, razão pela qual optou-se pela adoção do MySQL, um sistema gerenciador de banco de dados considerado versátil e com suporte a diversas plataformas e diferentes linguagens, bastante utilizado em aplicações web e desktop \cite{MySql:MySql}. Para facilitar a administração do banco de dados, a ferramenta MySQL Workbench \cite{MySql:WorkBench} também foi utilizada, visando prover auxílio na visualização dos dados do banco, realizar testes e gerenciar as mudanças.


\section{Apresentação da Plataforma} \label{sec:apresenta}
O \emph{LabInstru Web}, nome dado à plataforma desenvolvida no escopo deste trabalho de conclusão de curso, é o resultado da implementação das funcionalidades identificadas e prototipadas nas etapas anteriores, respeitando os modelos conceitual e de entidade-relacionamento propostos. Esta plataforma foi desenvolvida utilizando o \emph{framework} Web2py, banco de dados MySQL e tecnologias como JQuery e Bootstrap. A página inicial da aplicação encontra-se ilustrada na Figura \ref{fig:ap1}. A partir da página principal da aplicação, é possível aos seus usuários utilizarem as funcionalidades para autenticação no sistema, vide Figura \ref{fig:ap10}, bem como para recuperação de senha, vide Figura  \ref{fig:ap11}.

\begin{figure}[h!]
	\centering
	\includegraphics[width=0.9\textwidth]{./img/ap1.png}
	\caption{Página inicial da aplicação LabInstru Web. Fonte: Próprio autor.} \label{fig:ap1}
\end{figure}

\begin{figure}[h!]
	\centering
	\includegraphics[width=0.9\textwidth]{./img/ap10.png}
	\caption{Formulário de autenticação do sistema. Fonte: Próprio autor.} \label{fig:ap10}
\end{figure}



\begin{figure}[h!]
	\centering
	\includegraphics[width=0.9\textwidth]{./img/ap11.png}
	\caption{Formulário para recuperação de senha. Fonte: Próprio autor} \label{fig:ap11}
\end{figure}

A plataforma contempla dois perfis diferentes de acesso: um administrador, a ser desempenhado em termos práticos pela Profa. Maria Betânia Leal, responsável pelo LabInstru, e diversos usuários. O administrador, em especial, é o responsável por fornecer como entrada os dados advindos da estação meteorológica da EST para o LabInstru Web, conforme ilustrado na Figura \ref{fig:ap12}. A aplicação, após processar e persistir os dados advindos da estação, irá fornecer um sumário a respeito do status desta inserção, sob a forma de um \emph{log}. Este \emph{log} exibe quantas medições possuía o arquivo fornecido como entrada, quantas foram corretamente persistidas e quantas resultaram em falha. Um exemplo deste \emph{log} produzido pela aplicação é mostrado na Figura \ref{fig:ap13}.

\begin{figure}[h!]
	\centering
	\includegraphics[width=0.9\textwidth]{./img/ap12.png}
	\caption{Página que disponibiliza o formulário para inserção de medições. Fonte: Próprio autor} \label{fig:ap12}
\end{figure}

\begin{figure}[h!]
	\centering
	\includegraphics[width=0.9\textwidth]{./img/ap13.png}
	\caption{Exemplo de \emph{log} informando o resultado de uma inserção de medições. Fonte: Próprio autor} \label{fig:ap13}
\end{figure}



O administrador também fica responsável por cadastrar usuários, vide Figura \ref{fig:ap4}, que podem ser alunos de graduação e pós-graduação, outros pesquisadores, docentes, etc. Cabe também ao administrador da aplicação, por meio de uma listagem de usuários disponibilizada pela aplicação, realizar o gerenciamento dos usuários cadastrados na aplicação, conforme ilustrado na Figura \ref{fig:ap14}.

\begin{figure}[h!]
	\centering
	\includegraphics[width=0.9\textwidth]{./img/ap4.png}
	\caption{Página de cadastrado de usuário na plataforma LabInstru Web. Fonte: Próprio autor.} \label{fig:ap4}
\end{figure}

\begin{figure}[h!]
	\centering
	\includegraphics[width=0.9\textwidth]{./img/ap14.png}
	\caption{Lista de usuários cadastrados na aplicação. Fonte: Próprio autor.} \label{fig:ap14}
\end{figure}

Um menu superior, mostrado na Figura \ref{fig:ap1}, permite a autenticação dos usuários e também mostra as principais funcionalidades disponíveis na plataforma após realizada autenticação no sistema. Este menu é renderizado conforme o perfil do usuário. Por exemplo, o menu mostrado ao administrador dispõe da funcionalidade de apagar medições, funcionalidade não disponível no menu mostrado a um usuário. Os menus correspondentes aos diferentes perfis de usuário são ilustrados na Figura \ref{fig:ap8}. Ressalta-se que, independentemente do tipo do perfil de usuário, as principais funcionalidades da plataforma só poderão ser acessadas mediante prévia autenticação na plataforma via login e senha.

\todo{Nova versão da figura, refletindo o novo menu}
\begin{figure}[h!]
	\centering
	\includegraphics[width=0.9\textwidth]{./img/ap8.png}
	\caption{Exemplos dos menus para diferentes perfis de usuário. Fonte: Próprio autor.} \label{fig:ap8}
\end{figure}

As funcionalidades que permitem a mudança de senha e alteração de seus dados cadastrais são comuns aos dois perfis de usuários. Para ter acesso as mesmas, é necessário apenas que o usuário ou administrador encontre-se autenticado na aplicação. As Figuras \ref{fig:ap5} e \ref{fig:ap6} ilustram, respectivamente, os formulários para mudança de senha e alteração dos dados cadastrais de um determinado usuário (perfil).

\begin{figure}[h!]
	\centering
	\includegraphics[width=0.9\textwidth]{./img/ap5.png}
	\caption{Formulário para troca de senha. Fonte: Próprio autor.} \label{fig:ap5}
\end{figure}

\begin{figure}[h!]
	\centering
	\includegraphics[width=0.9\textwidth]{./img/ap6.png}
	\caption{Página referente a edição de usuário. Fonte: Próprio autor.} \label{fig:ap6}
\end{figure}


Os usuários da plataforma \emph{LabInstru Web} acessam os dados da estação meteorológica da EST por meio de consultas, nas quais fornecem datas iniciais e finais e escolhem as variáveis de interesse, conforme ilustrado na Figura \ref{fig:ap3}. Considerou-se como uma restrição essencial para garantir a integridade e a confiabilidade dos dados que apenas o administrador seria responsável por cadastrá-los.

\todo{Atualizar esta figura}
\begin{figure}[h!]
	\centering
	\includegraphics[width=0.9\textwidth]{./img/ap3.png}
	\caption{Página responsável por realizar as consultas. Fonte: Próprio Autor} \label{fig:ap3}
\end{figure}

Os resultados de uma consulta são exibidos em uma página apropriada, onde há opções para exportação dos mesmos nos formatos CSV, HTML, JSON, TSV e XML, conforme ilustra a Figura \ref{fig:ap7}.

\begin{figure}[h!]
	\centering
	\includegraphics[width=0.9\textwidth]{./img/ap7.png}
	\caption{Página responsável por exibir o resultado de uma consulta. Fonte: Próprio Autor} \label{fig:ap7}
\end{figure}


\subsection{Geração do Boletim Meteorológico}
Conforme identificado na elicitação de requisitos, uma das funcionalidades requeridas para a plataforma é a geração de boletins meteorológicos diários, detalhados anteriormente na Seção \ref{sec:boletim}.

Embora esta funcionalidade deva ser implementada em uma etapa posterior, conforme exposto na Seção \ref{sec:agenda}, foi identificada a necessidade de realizar uma especificação do mesmo junto ao cliente, elaborando uma visualização e estabelecendo os elementos que o mesmo deveria apresentar. Como resultado, tem-se o modelo apresentado na Figura \ref{fig:modeloBoletim}, que contempla a data, precipitação, temperaturas máxima e mínima, índice de calor, rajada e sua classificação na Escala de Beaufort, e ainda a logomarca do LabInstru e da Universidade do Estado do Amazonas.

\newpage

\begin{figure}[h!]
	\centering
	\includegraphics[width=0.9\textwidth]{./img/esbocoBoletim.png}
	\caption{Modelo referente ao boletim meteorológico gerado pelo LabInstru Web. Fonte: Próprio autor.} \label{fig:modeloBoletim}
\end{figure}

Produzir este modelo automaticamente a partir dos dados contidos na aplicação será uma das etapas posteriores deste trabalho.


\subsection{Suspensão da Funcionalidade de Geração de Gráficos}
Assim como no caso do boletim meteorológico, foi necessária uma discussão mais específica com o cliente a fim de elicitar os elementos do caso de uso ``Visualizar Gráficos'', descrito originalmente no Apêndice \ref{sec:aprendiceCasoUso}.

Embora o caso de uso tenha deixado claro qual o fluxo básico, não incluía a especificação de quais gráficos deveriam ser produzidos para os diferentes atributos, se um único gráfico com todas as informações ou se gráficos separados para cada atributo. Além deste aspecto, na discussão com a cliente foi ressaltado que um mesmo tipo de gráfico pode ser adequado em um cenário com uma certa quantidade de dados e inadequado para uma quantidade diferente, especialmente se muito grande, podendo vir a  prejudicar a visualização. Questionamentos acerca da escolha de cores, do tipo de gráfico, da amostragem dos dados para exibição, dentre outros, foram também levados em conta nesta reunião.

Como conclusão, cliente e desenvolvedor acordaram que a implementação desta funcionalidade poderia ser suspensa, por introduzir muitos aspectos que fogem ao propósito inicial do LabInstru Web e por haver ferramentas computacionais já usadas pelos profissionais do laboratório capazes de endereçar estas demandas. Esta decisão também implicou na ausência de necessidade de implementar a exportação de gráficos.

Este fato mostrou que, mesmo em uma etapa avançada do trabalho, os requisitos podem não estar suficientemente esclarecidos, sendo necessário revê-los e detalhá-los. Em particular, esta mudança não pôde ser antecipada pelo desenvolvedor por ter elencado inicialmente outros requisitos mais essenciais junto ao cliente, cuja implementação foi priorizada.


\subsection{Métricas de Software} \label{sec:metricas}
Algumas métricas foram obtidas do software produzido, auxiliando a compreender os esforços necessários ao seu desenvolvimento. Estas métricas são apresentadas a seguir:

\begin{itemize}
  \item Linhas de código: 7.264 no total, sendo:
  \begin{itemize}
    \item Arquivos Python: 1.714 linhas, contemplando conteúdo de linhas de código e comentários presentes em arquivos do tipo \texttt{.py};
    \item Arquivos HTML: 2.400 linhas, contidas em arquivos \texttt{.html} responsáveis por renderizar as \emph{views} aos usuários;
    \item Arquivos de estilo: 3.150 linhas, considerando os arquivos \texttt{.css} e comandos Javascript nele incluídos;
  \end{itemize}
  \item Funções disponíveis no \emph{Controller}: 28;
  \item Modelos de dados: 8, respeitando as modelagens conceitual e de entidade-relacionamento realizadas.
\end{itemize}

É importante salientar que algumas métricas tradicionais, a exemplo do número de classes, são inadequadas no contexto da utilização do \emph{framework} Web2py. O conceito análogo, neste cenário, é o de modelos de dados.  As funções no \emph{Controller}, por exemplo, podem também ser entendidas por meio de uma analogia com \emph{Web services}.

Outro aspecto que merece ser detalhado é número de linhas de código escritas na linguagem Python, em quantidade inferior aos demais tipos presentes. Embora em um primeiro momento possa-se ter a ideia errônea de que esta linguagem de programação teve um papel secundário ou terciário na aplicação produzida, cabe corrigir este pensamento recapitulando o poder de expressividade desta linguagem de programação, em que poucas linhas de código com sintaxe simples e de altíssimo nível podem ter uma semântica complexa, muitas vezes correspondendo à diversas linhas de código em outras linguagens de programação mais tradicionais. De fato, estima-se que, em comparação com as linguagens C, C++ e Java, os códigos produzidos em Python cheguem a ser de $50$ a $70\%$ menores \cite{Lutz:AprendendoPython}. Este baixo número também colabora para um menor dispêndio de tempo na leitura, entendimento e manutenção do código.


\section{Implantação da Solução Proposta} \label{sec:implantancao}
O Google App Engine\footnote{\url{https://cloud.google.com/appengine/}} foi a plataforma de computação em nuvem escolhida para fazer o \emph{deployment} do LabInstru Web. Ela foi desenvolvida pela Google em 2008 como uma tecnologia no modelo plataforma como serviço, virtualizando aplicações em múltiplos servidores, provendo hardware, conectividade, sistema operacional, dentre outros.

O Google App Engine foi escolhido para implantação do LabInstru Web por três razões principais: ($i$) o total suporte à linguagem de programação Python; ($ii$) o oferecimento de um nível de serviço gratuito compatível com o esperado de utilização pelos profissionais, pesquisadores e estudantes do LabInstru; ($iii$) alta disponibilidade e baixa latência, inclusive com servidores localizados no Brasil.

