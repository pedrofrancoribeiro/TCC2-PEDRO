

Com o intuito de mitigar as dificuldades enfrentadas por pesquisadores e estudantes do Laboratório de Instrumentação Meteorológica da Escola Superior de Tecnologia da Universidade do Estado do Amazonas no tocante à manipulação dos dados meteorológicos coletados pela Estação Meteorológica Automática da EST, este trabalho de conclusão de curso teve por objetivo projetar e implementar uma plataforma web, denominada LabInstru Web, com vistas a contribuir para automatização e simplificação do armazenamento, gerenciamento, manutenção e disponibilização dos dados produzidos por esta estação, colaborando também na automatização da geração de boletins meteorológicos voltados para divulgação das medições junto à comunidade em geral. A plataforma desenvolvida permite a manutenção de dois tipos de registros produzidos pela estação meteorológica da EST, a consulta com diferentes parâmetros, a exportação de dados em diferentes formatos, a consulta de um calendário de disponibilidade de medições, a geração automática de boletins meteorológicos diários e a manutenção de dois perfis distintos de usuário, funcionalidades identificadas a partir das necessidades reais dos pesquisadores do Labotarório. Desenvolvido segundo o Processo Ágil Unificado, o LabInstru Web foi implementado na linguagem de programação Python com a utilização do \emph{framework full-stack} Web2py, além de tecnologias como MySQL, JQuery e Bootstrap. A solução desenvolvida foi implantada com o Google App Engine.


\noindent \textbf{Palavras-Chave}: Plataforma Web; Python; Dados Meteorológicos.
