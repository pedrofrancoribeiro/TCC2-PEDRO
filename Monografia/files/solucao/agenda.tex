Considerando a Atividade 5 apresentada na Metodologia deste trabalho, vide Seção \ref{sec:metodologia}, uma das atividades necessárias consistia em elencar uma ordem de implementação dos requisitos, considerando as dependências entre os mesmos e o cronograma disponível do trabalho de conclusão de curso.

Após a descoberta dos requisitos e da documentação dos mesmos em diagramas de caso de uso, considerou-se então a seguinte agenda para implementação:

\begin{enumerate}
  \item Escopo do Trabalho de Conclusão de Curso I: Módulo Gerencia Conta de Usuário, Módulo Usuário, Módulo Gerencia Medições e consulta de medições do Módulo Consulta Medições;
  \item Escopo do Trabalho de Conclusão de Curso II: demais funcionalidades remanescentes no Módulo Consulta Medições.
\end{enumerate}

É interessante observar que todas as funcionalidades programadas para o Trabalho de Conclusão de Curso I são pré-requisitos para as funcionalidades a serem desenvolvidas no Trabalho de Conclusão de Curso II.
