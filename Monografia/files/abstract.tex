
In order to mitigate the difficulties faced by researches and students from the Laboratory of Meteorological Instrumentation  at Superior School of Technology from Amazonas State University regarding data manipulation produced by the EST automated weather station, this work aimed to design and develop a web platform, called LabInstru Web, whose objective is to   contribute to automation and simplification of the storage, management, maintenance and availability of the data produced by this station, also collaborating in the generation of meteorological bulletins aimed at dissemination of measurements with the community in general. The developed platform allows the maintenance of two types of records produced by the EST weather station, searches with different parameters, export of data in different formats, consultation of a calendar of availability of measurements, automatic generation of daily meteorological bulletins and the maintenance of two distinct user profiles, functionalities identified from the real needs of the researchers of the label. Developed according to Agile Unified Process, LabInstru Web was implemented in Python with the support of the full-stack framework Web2py, altogether with technologies such as MySQL, JQuery and Bootstrap. The deployment was carried out with Google App Engine.

\noindent \textbf{Keywords}: Web platform; Python; Meteorologic data.
