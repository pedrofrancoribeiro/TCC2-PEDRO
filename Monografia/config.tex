\usepackage[utf8]{inputenc}
\usepackage[T1]{fontenc}
\usepackage{ae}
\usepackage[brazil]{babel}
\usepackage{a4wide}
\usepackage{comment}
\usepackage[pdftex]{graphicx,color}
\usepackage{graphics}
\usepackage{cite}
\usepackage{longtable}
\usepackage{float}
\usepackage{fancyvrb}
\usepackage{fancyhdr}
\usepackage{setspace}
\usepackage{amsmath, amsfonts, amssymb}
\usepackage{lscape}
\usepackage{textcase}
\usepackage{anysize}
\usepackage{setspace}
\usepackage{booktabs}
\usepackage{url}
\usepackage{subfig}
\usepackage{cite}
\usepackage[alf]{abntex2cite}
\usepackage{courier}
\usepackage{xcolor}
\usepackage{siunitx}
\usepackage{framed}
\usepackage{verbatim}
\usepackage{hhline}
\usepackage{trivfloat}
\usepackage{enumitem}

\linespread{1.1}
\trivfloat{quadro}
\floatstyle{plaintop} %Força posição da leganda para o topo
\restylefloat{quadro} %Força posição da leganda para o topo
\renewcommand{\listquadroname}{Lista de quadros} %Forçar texto na lista de quadros

\usepackage{indentfirst}
\usepackage{lineno}
\usepackage{listings}

\marginsize{20mm}{20mm}{20mm}{15mm}


%% Cabeçalhos
\renewcommand{\topfraction}{1}
\renewcommand{\bottomfraction}{1}
\renewcommand{\floatpagefraction}{1}
\renewcommand{\textfraction}{0}
%\renewcommand{\baselinestretch}{2}
\doublespacing %espa�amento duplo
\sloppy

%% Nomes
\floatstyle{plain}  %%% tipos: plain, boxed, ruled
\newfloat{codigo}{tbp}{lop}[section]
\floatname{codigo}{Código}

%%% nome para ser usado no sumrio

%\newcommand{\listofcodename}{Lista de C\'{o}digos}



% RESUMO ----------------------------------------------------------------------------------------------------------------------------------------------------------------------

\newcommand{\resumo}[1]{
\begin{center} \LARGE \bf Resumo \end{center}

\vskip 4em
\input{#1}

\newpage

}

% ABSTRACT ----------------------------------------------------------------------------------------------------------------------------------------------------------------------

\newcommand{\abstractt}[1]{
\begin{center} \LARGE \bf Abstract \end{center}

\vskip 4em
\input{#1}

\newpage

}

% Sumário -----------
\newcommand{\sumario}{
\renewcommand{\contentsname}{Sum\'{a}rio}
\tableofcontents
\addcontentsline{toc}{chapter}{\listtablename}
\listoftables

\newpage
\addcontentsline{toc}{chapter}{\listfigurename}
\listoffigures
%\addcontentsline{toc}{chapter}{\listofcodename}
%\listof{codigo}{\listofcodename}  % Lista de Códigos

\clearpage
}
