Assim como no caso do boletim meteorológico, foi necessária uma discussão mais específica com o cliente a fim de elicitar os elementos do caso de uso ``Visualizar Gráficos'', descrito originalmente no Apêndice \ref{sec:aprendiceCasoUso}.

Embora o caso de uso tenha deixado claro qual o fluxo básico, não incluía a especificação de quais gráficos deveriam ser produzidos para os diferentes atributos, se um único gráfico com todas as informações ou se gráficos separados para cada atributo. Além deste aspecto, na discussão com a cliente foi ressaltado que um mesmo tipo de gráfico pode ser adequado em um cenário com uma certa quantidade de dados e inadequado para uma quantidade diferente, especialmente se muito grande, podendo vir a  prejudicar a visualização. Questionamentos acerca da escolha de cores, do tipo de gráfico, da amostragem dos dados para exibição, dentre outros, foram também levados em conta nesta reunião.

Como conclusão, cliente e desenvolvedor acordaram que a implementação desta funcionalidade poderia ser suspensa, por introduzir muitos aspectos que fogem ao propósito inicial do LabInstru Web e por haver ferramentas computacionais já usadas pelos profissionais do laboratório capazes de endereçar estas demandas. Esta decisão também implicou na ausência de necessidade de implementar a exportação de gráficos.

Este fato mostrou que, mesmo em uma etapa avançada do trabalho, os requisitos podem não estar suficientemente esclarecidos, sendo necessário revê-los e detalhá-los. Em particular, esta mudança não pôde ser antecipada pelo desenvolvedor por ter elencado inicialmente outros requisitos mais essenciais junto ao cliente, cuja implementação foi priorizada.
