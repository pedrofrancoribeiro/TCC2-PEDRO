\chapter{Considerações Finais} \label{cap:consideracoes}

Tendo como motivação a mitigação das dificuldades diárias enfrentadas pelos pesquisadores e alunos do Laboratório de Instrumentação Meteorológica da Escola Superior de Tecnologia da Universidade do Estado do Amazonas para manipulação e visualização dos dados da Estação Meteorológica da EST, este trabalho propôs o \emph{LabInstru Web}, uma plataforma web que com vistas a contribuir para automatização e simplificação do armazenamento, gerenciamento, manutenção e disponibilização dos dados produzidos por esta estação, colaborando também na automatização da geração de boletins meteorológicos voltados para divulgação das medições junto à comunidade em geral.

Desenvolvido segundo o Processo Ágil Unificado, o LabInstru Web foi implementado na linguagem Python utilizando o \emph{framework web full-stack} Web2py. Para o \emph{front-end}, as tecnologias ágeis JQuery e Bootstrap foram cruciais para a produção de resultados visualmente agradáveis junto aos usuários, respeitando as especificações feitas anteriormente junto à cliente. Quanto ao \emph{back-end}, o sistema gerenciador de banco de dados MySQL foi utilizado para persistência por prover um bom desempenho na manipulação de uma quantidade significativa de dados. A implantação da plataforma foi feita com o Google App Engine, por ser totalmente compatível com a linguagem Python e com as tecnologias utilizadas no desenvolvimento, bem como pelo oferecimento de uma alta disponibilidade e baixa latência.

O desenvolvimento deste trabalho de conclusão de curso proporcionou a aplicação de diversos conhecimentos obtidos nas disciplinas do curso de Engenharia de Computação com vistas a resolver um problema de um contexto real. Em particular, os conhecimentos adquiridos nas disciplinas de Engenharia de Software, Banco de Dados e Linguagem de Programação foram mais predominantes para o desenvolvimento deste trabalho. Vale salientar que, com o suporte oferecido pela formação em Engenharia de Computação, foi possível adquirir conhecimentos específicos e tecnológicos para vencer algumas limitações enfrentadas ao longo do projeto, especialmente no tocante ao desenvolvimento da interface com o usuário com a tecnologia Bootstrap.

Alguns desdobramentos futuros deste trabalho decorrem da utilização pelos seus usuários, com os quais pode-se identificar novas funcionalidades a serem desenvolvidas, refinar requisitos já implementados e avaliar a performance da solução implantada. Outros caminhos possíveis consistem para melhoria do LabInstru web podem considerar também a integração de trabalhos desenvolvidas pelas orientadores com \emph{machine learning} para previsão e descrição dos dados sobre a precipitação em Manaus.
