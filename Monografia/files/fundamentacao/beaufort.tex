A Escala de Beaufort é uma tabela criada pelo meteorologista anglo-irlandês Francis Beaufort no inicio do século XIX que tem como objetivo classificar a intensidade dos ventos, tendo como parâmetros a velocidade dos ventos e os efeitos resultantes das ventanias no mar e em terra. Segundo a Escala de Beaufort, por exemplo, ventos na escala 8 são capazes de mover as folhas das árvores e inicializar os trabalhos de moinhos (efeitos em terra), no mar, resulta uma ligeira ondulação sem rebentação (aspecto do mar) \cite{MetOffice:Beaufort}. Em 2007 \cite{Sentelhas:EscalaBeaufort} adaptou a escala de Beaufort para níveis continentais, a escala de Beaufort adaptada é ilustrada na Tabela \ref{tab:beaufort}.

\newpage

\begin{table}[h!]


  \caption{Escala de Beaufort. Fonte: \cite{Sentelhas:EscalaBeaufort}} \label{tab:beaufort}
  \centering
\begin{tabular}{ccc}
 \toprule
 \textbf{Escala} & \textbf{Classificação} & \textbf{Velocidade (m/s)}\\
 \midrule
0 & Calmo & $<$ 0,5\\
1 & Quase calmo & 0,5 a 1,5\\
2 & Brisa amena & 1,6 a 2,9\\
3 & Vento Leve & 3,0 a 5,7\\
4 & Vento Moderado & 5,8 a 8,3\\
5 & Vento Forte & 8,4 a 11,1\\
6 & Vento Muito Forte &  11,2 a 13,9\\
7 & Vento fortíssimo & 14,0 a 16,6\\
8 & Ventania & 16,7 a 20,9\\
9 & Vendaval & 21,0 a 27,8\\
10 & Tornado, furacão & $>$ 27,8\\
\bottomrule
\end{tabular}
\end{table}

Saber especificar a escala de uma determinada medição para a velocidade do vento utilizando a escala de Beaufort e consultar os efeitos que esse tipo de vento pode ocasionar sobre o ambiente em que ocorreu são informações comumente divulgadas em boletins meteorológicos para a população em geral, sendo úteis, inclusive, na emissão de alertas sobre  ventanias fortes, tempestades e até mesmo furacões. 
